\documentclass[12pt,letter]{article}

\usepackage[margin=1.0in]{geometry}

\begin{document}
\noindent\textsc{\large CS 314 Exam One Review --- Insert All To Front}

\vspace{6pt}
\noindent\textbf{Array-based Lists}

\vspace{2pt}
\noindent Implement an instance method for the \texttt{GenericList} class which will 
insert all of the elements from another \texttt{GenericList} to the front of \texttt{this} list.
The \texttt{GenericList} argument will remain unchanged after this operation.

\vspace{4pt}
\noindent Complete the following method.
\begin{verbatim}
  // Removes inserts all elements from other into this
  // pre: other != null
  // post: other is unchanged
  public void insertAllFront(GenericList<E> other)
\end{verbatim}

\vspace{4pt}

\noindent Here are some example calls to \texttt{insertAllFront()}:
\newline
\noindent \texttt{4, 5, 6].insertAllFront([1, 2, 3])} $\rightarrow$ \texttt{this = [1, 2, 3, 4, 5, 6]}
\newline
\noindent \texttt{[4].insertAllFront([3, 1])} $\rightarrow$ \texttt{this = [3, 1, 4]}
\newline
\noindent \texttt{[2, 3, 4, 5].insertAllFront([1])} $\rightarrow$ \texttt{this = [1, 2, 3, 4, 5]}
\newline
\noindent \texttt{[].insertAllFront([1, 2, 3])} $\rightarrow$ \texttt{this = [1, 2, 3]}
\newline
\noindent \texttt{[3, 1, 1].insertAllFront([])} $\rightarrow$ \texttt{this = [3, 1, 1]}
\newline

\noindent Your method will be in the following \texttt{GenericList} class:

\begin{verbatim}
  public class GenericList<E>{
    private int size;
    private E[] con;
    // ...
  }

\end{verbatim}

\noindent \textbf{Do not use or assume there are any provided methods in the GenericList class.}

\noindent \textbf{You may create a new internal array container.}

\noindent \textbf{Do not use any other Java classes or methods.}

\clearpage
\begin{verbatim}
  // Removes inserts all elements from other into this
  // pre: other != null
  // post: other is unchanged
  public void insertAllFront(GenericList<E> other){
\end{verbatim}

\end{document}