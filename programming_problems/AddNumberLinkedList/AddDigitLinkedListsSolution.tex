\documentclass[12pt,letter]{article}

\usepackage[margin=1.0in]{geometry}

\begin{document}
\noindent\textsc{\large CS 314 Final Review --- Add to Digits LinkedList --- \textbf{Solution}}

\vspace{6pt}
\noindent\textbf{Linked Lists}

\vspace{2pt}
\noindent Suppose you are given a LinkedList where each node represents a digit in a positive integer. The head of the list holds the most significant digit
of the number while the tail of the list holds the least significant. This list of digits will be represented using the \texttt{DigitLinkedList} class.

\vspace{2pt}
\noindent Write an instance method for the \texttt{DigitLinkedList} class which will add an arbitrary, positive integer to the number represented by
the list.

\vspace{2pt}
\noindent It is guaranteed that \texttt{this} is in a valid state when \texttt{addValue()} is called. This means that each \texttt{DigitNode}'s value will
be between 0 and 9 inclusive. It also means that \texttt{first != null} (if the list stores the the value zero, then it would have a single \texttt{DigitNode} with
a \texttt{digit} value of 0).


\noindent Examples of calls to \texttt{addValue()}. The result shown is the new state of \texttt{this}.
\begin{itemize}
  \item \texttt{[3, 1, 1].addValue(3) => [3, 1, 4]}
  \item \texttt{[3, 1, 1].addValue(20) => [3, 3, 1]}
  \item \texttt{[4, 2, 9].addValue(10) => [4, 3, 9]}
  \item \texttt{[0].addValue(408) => [4, 0, 8]}
  \item \texttt{[1, 1].addValue(303) => [3, 1, 4]}
\end{itemize}

\vspace{6pt}
\noindent Use the follwing \texttt{DigitLinkedList} implementation.
\begin{verbatim}
public class DigitLinkedList {
    private DigitNode first;

    private static class DigitNode {
        // The nested DigitNode class.
        private int digit;
        private DigitNode next;
        private DigitNode(int d, DigitNode n){
          digit = d;
          next = n;
        }
    }
}
\end{verbatim}

\noindent You may create new \texttt{DigitNode} objects using the provided constructor.

\noindent \textbf{You may not create any other data structures.}

\noindent Do not use any other Java classes or methods.

\clearpage
\begin{verbatim}
/* Pre:  val >= 0; this list is in a valid state -> first != null
 * Post: this list will represent the sum of the previous state + val
 *       this list will still be in a valid state
 */
public void addVal(int value){
    // Once we've updated the existing nodes, we may need to 
    // create new nodes if the number we create is too large 
    // to fit in the number of nodes we had in the beginning
    int overflow = helper(first, value);
    while(overflow > 0){
        first = new DigitNode(overflow % 10, first);
        overflow /= 10;
    }
}   

private int helper(DigitNode n, int valToAdd){
    int carry = valToAdd;
    if(n.next != null){
        carry = helper(n.next, valToAdd);
    }
    int sum = n.digit + carry;
    n.digit = sum % 10;
    return sum / 10;
}
\end{verbatim}
\end{document}
