\documentclass[12pt,letter]{article}

\usepackage[margin=1.0in]{geometry}

\begin{document}
\noindent\textsc{\large CS 314 Final Review --- Hash Table Contains}

\vspace{6pt}
\noindent\textbf{Hash Tables}

\vspace{2pt}
\noindent Write an instance method for the provided \texttt{HashTable314} class which will determine whether a given element is present in the table.
This hash table implementation uses linked list buckets to handle hash collisions. These linked lists are \textbf{not} kept in sorted order.
Elements in the hash table are placed in the bucket corresponding to their hash code modulo the length of the array.

\vspace{4pt}
\noindent \emph{This problem is about Hash Tables which use buckets for collisions. If you would like practice with a Hash Table which uses linear probing, 
revisit the extra section problem from Nov. 25}

\vspace{4pt}

\begin{verbatim}
/* Pre: val != null
 * Post: returns true iff val is contained in the hash table.
 *       The hash table is unchanged as a result of this operation.
 */
 public boolean contains(E val);
\end{verbatim}

\vspace{6pt}
\noindent You may use the following \texttt{HashTable314} implementation.
\begin{verbatim}
public class HashTable314<E> {
    private static final int INITIAL_CAPACITY = 10;

    private BucketNode<E>[] con;
    private int size;

    private static class BucketNode<E> {
      private E data;
      private BucketNode<E> next;
    }
}
\end{verbatim}

\noindent You may not use or assume any other methods exist in the \texttt{HashTable314} class.
\noindent Do not use any other Java classes or methods.

\clearpage
\begin{verbatim}
  /* Pre: val != null
  * Post: returns true iff val is contained in the hash table.
  *       The hash table is unchanged as a result of this operation.
  */
  public boolean contains(E val) {
\end{verbatim}
\end{document}
