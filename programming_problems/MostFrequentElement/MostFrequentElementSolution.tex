\documentclass[12pt,letter]{article}

\usepackage[margin=1.0in]{geometry}

\begin{document}
\noindent\textsc{\large CS 314 Final Review --- Most Frequent Element --- \textbf{Solution}}

\vspace{6pt}
\noindent\textbf{Maps}

\vspace{2pt}
\noindent Implement an instance method which, for a given list of elements returns the element which appears in the list the most frequently. 
This is also known as the \emph{mode}. This method will be a part of the \texttt{List314} class, 
an array-based list similar to the one created in lecture. Return the element in the list which is 
present most often, if there is a tie, return the element which occurs first in the list.
The list will not be altered as a result of this call.

\vspace{4pt}
\noindent Complete the following method.
\begin{verbatim}
  // Returns the most frequent element in the list
  // pre: list != null, list.size() > 0
  // post: returns the most frequent element, list is unchanged
  public E mostFrequent()
\end{verbatim}

\vspace{4pt}
\noindent Here are some example calls to \texttt{mostFrequent()}:
\newline
\noindent \texttt{["A", "A", "B", "A", "B"].mostFrequent())} $\rightarrow$ \texttt{"A"}
\newline
\noindent \texttt{["A", "B", "C", "A", "B", "A", "BB"].mostFrequent()} $\rightarrow$ \texttt{"A"}
\newline
\noindent \texttt{[1, 2, 3, 2, 1].mostFrequent()} $\rightarrow$ \texttt{1}
\newline
\noindent \texttt{[3, 1, 4].mostFrequent()} $\rightarrow$ \texttt{3}
\newline

\noindent Your method will be in the following \texttt{List314} class:

\begin{verbatim}
  public class List314<E>{
    private int size;
    private E[] con;
    // ...
  }

\end{verbatim}

\noindent You may create a \texttt{TreeMap} or \texttt{HashMap}. 
\newline
\noindent \textbf{Do not use any other Java classes or methods.}

\clearpage
\begin{verbatim}
// Returns the most frequent element in the list
// pre: list != null, list.size() > 0
// post: returns the most frequent element, list is unchanged
public E mostFrequent(){
    HashMap<E, Integer> freqs = new HashMap<E, Integer>();
    E mostSeen = null;
    int maxFreq = -1;
    for(int i = 0; i < size; i++){
        int freq = freqs.getOrDefault(con[i], 0);
        freq++;
        freqs.put(con[i], freq);
        if(freq > maxFreq){
            mostSeen = con[i];
            maxFreq = freq;
        }
    }
    return mostSeen;
}
\end{verbatim}

Relatively straightforward map and ArrayList problem. Using a map for this problem makes keeping
track of the frequencies of the elements easy and efficient. However, using a TreeMap for this problem would be incorrect because
TreeMaps require elements whose class implements the \texttt{Comparable} interface, and since this was never guaranteed
in a precondition, using a TreeMap could result in a \texttt{ClassCastException}. 

Also, keeping track of the maximum frequency we've seen so far as we traverse the array makes our lives a lot easier. It prevents us from
having to go through the map once we're done to find the max and it makes it easy to break ties.
\end{document}
