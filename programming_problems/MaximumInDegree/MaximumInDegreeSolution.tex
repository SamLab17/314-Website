\documentclass[12pt,letter]{article}

\usepackage[margin=1.0in]{geometry}

\begin{document}
\noindent\textsc{\large CS 314 Final Review --- Maximum In-Degree}

\vspace{6pt}
\noindent\textbf{Graphs}

\vspace{2pt}
\noindent Write an instance method for the provided \texttt{Graph} class which finds the
maximum in-degree out of all the vertices in the graph. The in-degree of a vertex
is the number of edges which are directed into it.

\vspace{4pt} 
\noindent The graph for this problem is directed and stored using adjacency lists.

\vspace{6pt}
\noindent Complete the following instance method.
\begin{verbatim}
  // Finds the maximum in-degree in the graph
  // pre: vertices.size() > 0
  // post: the structure of the graph is not altered by this operation
  public int maxInDegree() {
\end{verbatim}

\vspace{4pt}
\noindent Use the following Graph implementation.

\begin{verbatim}
  public class Graph {
    // The vertices in the graph.
    private Map<String, Vertex> vertices;

    // Sets scratch to 0 for all vertices
    private void clearAll(){ /* ... */ }

    private static class Vertex {
      private String name;
      private List<Edge> adjacent;
      private int scratch;
    }

    private static class Edge {
      private Vertex dest;
    }    

  }
\end{verbatim}

\noindent \textbf{You may not create any new data structures. \newline Do not use any other Java classes or methods.}

\clearpage
\begin{verbatim}
  // Finds the maximum in-degree in the graph
  // pre: vertices.size() > 0
  // post: the structure of the graph is not altered by this operation
  public int maxInDegree() {
    clearAll();
    for(Vertex v : vertices.values()){
      for(Edge e : v.adjacent){
        Vertex neighbor = e.dest;
        neighbor.scratch++;
      }
    }

    int max = 0;
    for(Vertex v : vertices.values())
      if(v.scratch > max)
        max = v.scratch;
  
    return max;
  }
\end{verbatim}
\end{document}
