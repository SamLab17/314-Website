\documentclass[12pt,letter]{article}

\usepackage[margin=1.0in]{geometry}

\begin{document}
\noindent\textsc{\large CS 314 Final Review --- Hash Table Resize}

\vspace{6pt}
\noindent\textbf{Hash Tables}

\vspace{2pt}
\noindent Write a private instance method \texttt{resize} for the provided \texttt{HashTable314} class which will double the size of the internal
array and re-index the elements in the hash table. This hash table implementation uses linked bucket nodes to handle hash collisions. 
These linked lists are \textbf{not} kept in sorted order. Elements in the hash table are placed in the bucket corresponding to their 
hash code modulo the length of the array.

\vspace{4pt}
\noindent Your \texttt{resize} method should also return the magnitude (absolute value) of the largest change in index as a result of the resize.
For example, if an element was originally at index $1$ in \texttt{con} but after a call to \texttt{resize} its index became $11$, that is a 
change in index of $10$. Your method will return the largest of these changes.

\vspace{4pt}
\noindent \emph{This problem is about Hash Tables which use buckets for collisions. If you would like practice with a Hash Table which uses linear probing, 
look at the last problem on the Spring 2019 final.}

\vspace{4pt}

\begin{verbatim}
/* Pre: none
 * Post: con will be twice as large, size is unchanged
 */
 private int resize();
\end{verbatim}

\vspace{6pt}
\noindent You may use the following \texttt{HashTable314} implementation.
\begin{verbatim}
public class HashTable314<E> {
    private static final int INITIAL_CAPACITY = 10;

    private BucketNode<E>[] con;
    private int size;

    private static class BucketNode<E> {
      private E data;
      private BucketNode<E> next;
    }
}
\end{verbatim}


\noindent You may use the absolute value method (\texttt{abs}) from the \texttt{Math} class.

\noindent You may not use or assume any other methods exist in the \texttt{HashTable314} class.

\noindent Do not use any other Java classes or methods.

\clearpage
\begin{verbatim}
 /* Pre: none
  * Post: con will be twice as large, size is unchanged
  */
  private int resize(){
 \end{verbatim}
\end{document}
