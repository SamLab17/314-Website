\documentclass[12pt,letter]{article}
\usepackage{tikz-qtree}

\usepackage[margin=1.0in]{geometry}

\begin{document}
\noindent\textsc{\large CS 314 Final Review --- Linked List Sums}

\vspace{6pt}
\noindent\textbf{Linked Lists \& Recursion}

\vspace{2pt}
\noindent Implement an instance method for the singly-linked \texttt{IntLinkedList} class which changes a nodes value to be the sum of its own value
plus the value of all subsequent nodes in the list.

\vspace{4pt}
\noindent Complete the following method.
\begin{verbatim}
  // Changes each node's int value to be the sum of itself plus the sum of
  // all the values which come after it in the list.
  // pre: size > 0
  // post: size is unchanged
  public void addValuesToRight() {
\end{verbatim}

\vspace{4pt}
\noindent Here are some example calls to \texttt{addValuesToRight()}:
\begin{verbatim}
    [1, 1, 1, 1, 1].addValuesToRight() => [5, 4, 3, 2, 1]
    [1, 2, 3, 4, 5].addValuesToRight() => [15, 14, 12, 9, 5]
    [0, 0, 0, 0, 2].addValuesToRight() => [2, 2, 2, 2, 2]
    [5, 1, 2, 0].addValuesToRight() => [8, 3, 2, 0]
    [3, 1, 4].addValuesToRight() => [8, 5, 4]
    [7].addValuesToRight() => [7]
\end{verbatim}

\vspace{4pt}
\noindent You may use the following \texttt{IntLinkedList} implementation:

\begin{verbatim}
  public class IntLinkedList {
    IntNode first;
    int size;

    //Nested node class
    private static class IntNode{
      IntNode next;
      int value;
    }
  }
\end{verbatim}

\noindent Your solution should be as efficient as possible in terms of time.

\noindent You may create helper methods.

\noindent \textbf{Do not create any new data structures or use any other Java classes or methods.}

\clearpage
\begin{verbatim}
    // Changes each node's int value to be the sum of itself plus the sum of
    // all the values which come after it in the list.
    // pre: size > 0
    // post: size is unchanged
    public void addValuesToRight() {
\end{verbatim}
\end{document}
