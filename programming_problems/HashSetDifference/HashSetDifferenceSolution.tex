\documentclass[12pt,letter]{article}

\usepackage[margin=1.0in]{geometry}

\begin{document}
\noindent\textsc{\large CS 314 Final Review --- Hash Set Difference -- \textbf{Solution}}

\vspace{6pt}
\noindent\textbf{Hash Tables/Sets}

\vspace{2pt}
\noindent Write an instance method for the provided \texttt{HashSet314} class which will find the difference of this set with an \texttt{otherSet} 
parameter. This method will return a new \texttt{HashSet314} object and will leave both \texttt{this} and \texttt{otherSet} unaltered. 

\vspace{2pt}

\noindent The \texttt{HashSet314} class uses a hash table as its internal storage container. There are no duplicate elements are in the set.
The internal array for the hash table uses linear probing to handle hash collisions. If an element which was previously in the set was removed, its spot
in the array will have been replaced with a refernce to \texttt{EMPTY}, a static, empty \texttt{Object}.

\vspace{2pt}

\noindent When creating the new set, you can assume the \texttt{HastSet314} class's \texttt{add(E val)} method (and all necessary resizing/rehashing) has been implemented correctly.

\vspace{2pt}

\noindent An element will be present in the resulting set if and only if it is present in \texttt{this} but not in \texttt{otherSet}.

\vspace{4pt}

\vspace{6pt}
\noindent You may use the following \texttt{HashSet314} implementation.
\begin{verbatim}
public class HashSet314<E> {
    private static final int INITIAL_CAPACITY = 10;
    private static final Object EMPTY = new Object();

    // All non-null elements in this array are guaranteed to 
    // be either EMPTY or of type E.
    private Object[] con;
    private int size;

    //HashSet constructor
    public HashSet314(){
      con = (E[]) new Object[INITIAL_CAPACITY];
    }

    //You may use this method, assume it has been implemented here
    public boolean add(E val);
}
\end{verbatim}

\vspace{4pt}
\noindent Notice that the internal array of \texttt{HashSet314} is of type \texttt{Object[]}. This is necessary to hold on to the 
reference to \texttt{EMPTY}, but you can safely assume that all non-\texttt{null} and non-\texttt{EMPTY} reference stored in the array
will be to objects of type \texttt{E}. (This may result you in making some casts which a compiler would typically call unsafe, but it is mostly
unavoidable in this situation).

\vspace{4pt}

\noindent You may not use methods in the \texttt{HashSet314} class except for \texttt{add(E val)} and the constructor.
\noindent You may use the \texttt{hashCode()} method and methods from the \texttt{Math} class, but do not use any other Java classes or methods.
\clearpage
\begin{verbatim}
 /* Pre: otherSet != null
  * Post: returns a new set which represents this - otherSet.
  *       Both this and otherSet are unaltered by this method call.
  */
public HashSet314<E> difference(HashSet314<E> otherSet){
    HashSet314<E> result = new HashSet314<>();
    int elementsSeen = 0;
    for(int i = 0; i < con.length && elementsSeen < size; i++) {
        if(con[i] != null && con[i] != EMPTY){
            elementsSeen++;
            if(!otherSet.contains(con[i]))
                result.add((E) con[i]); //Gacky cast, but necessary
        }
    }
    return result;
}

//New, private instance method to check if val is contained in a set.
//We made this private because we would want the publicly-accessible 
//contains() method to only accept vals of type E
private boolean contains(Object val){
    int index = Math.abs(val.hashCode() % con.length);
    int numVisited = 0;
    while(con[index] != null && !con[index].equals(val) 
        && numVisited < con.length){

        numVisited++;
        index = (index + 1) & con.length;
    }
    //Be careful of nulls here
    return val.equals(con[index]);
}
\end{verbatim}

\noindent This question requires you to deal with two different hash sets, where indexes of elements in one set
are most likely not the same as the in the other set. The hardest part of this problem is checking to see 
if an element is in the other set. We have to first find the index we would expect the value to occur by taking
the absolute value of \texttt{hashCode() \% con.length}. Then, since the hash set uses linear probing, we may
have to iterate a few times down the array before we find the element of interest. To be more specific, we have to keep
checking subsequent indices in the array until: we've found the element, we hit a null (this signifies the end of a block of elements
with the same hash code), or we've visited every element in the array (no null spots, element not present).
\vspace{4pt}

\noindent In the for loop of the \texttt{difference} method, there's an additional condition to stop looping: if we've seen every 
element in the hash set before we've gone through the entire array. If this problem appeared on an exam, not having this early
stopping condition would almost certainly result in losing points.
\end{document}
