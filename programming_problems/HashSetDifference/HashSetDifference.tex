\documentclass[12pt,letter]{article}

\usepackage[margin=1.0in]{geometry}

\begin{document}
\noindent\textsc{\large CS 314 Final Review --- Hash Set Difference}

\vspace{6pt}
\noindent\textbf{Hash Tables/Sets}

\vspace{2pt}
\noindent Write an instance method for the provided \texttt{HashSet314} class which will find the difference of this set with an \texttt{otherSet} 
parameter. This method will return a new \texttt{HashSet314} object and will leave both \texttt{this} and \texttt{otherSet} unaltered. 

\vspace{2pt}

\noindent The \texttt{HashSet314} class uses a hash table as its internal storage container. There are no duplicate elements are in the set.
The internal array for the hash table uses linear probing to handle hash collisions. If an element which was previously in the set was removed, its spot
in the array will have been replaced with a refernce to \texttt{EMPTY}, a static, empty \texttt{Object}.

\vspace{2pt}

\noindent When creating the new set, you can assume the \texttt{HastSet314} class's \texttt{add(E val)} method (and all necessary resizing/rehashing) has been implemented correctly.

\vspace{2pt}

\noindent An element will be present in the resulting set if and only if it is present in \texttt{this} but not in \texttt{otherSet}.

\vspace{4pt}

\vspace{6pt}
\noindent You may use the following \texttt{HashSet314} implementation.
\begin{verbatim}
public class HashSet314<E> {
    private static final int INITIAL_CAPACITY = 10;
    private static final Object EMPTY = new Object();

    // All non-null elements in this array are guaranteed to 
    // be either EMPTY or of type E.
    private Object[] con;
    private int size;

    //HashSet constructor
    public HashSet314(){
      con = (E[]) new Object[INITIAL_CAPACITY];
    }

    //You may use this method, assume it has been implemented here
    public boolean add(E val);
}
\end{verbatim}

\vspace{4pt}
\noindent Notice that the internal array of \texttt{HashSet314} is of type \texttt{Object[]}. This is necessary to hold on to the 
reference to \texttt{EMPTY}, but you can safely assume that all non-\texttt{null} and non-\texttt{EMPTY} reference stored in the array
will be to objects of type \texttt{E}. (This may result you in making some casts which a compiler would typically call unsafe, but it is mostly
unavoidable in this situation).

\vspace{4pt}

\noindent You may not use methods in the \texttt{HashSet314} class except for \texttt{add(E val)} and the constructor.
\noindent You may use the \texttt{hashCode()} method and methods from the \texttt{Math} class, but do not use any other Java classes or methods.

\clearpage
\begin{verbatim}
  /* Pre: otherSet != null
  * Post: returns a new set which represents this - otherSet.
  *       Both this and otherSet are unaltered by this method call.
  */
  public HashSet314<E> difference(HashSet314<E> otherSet){
\end{verbatim}
\end{document}
