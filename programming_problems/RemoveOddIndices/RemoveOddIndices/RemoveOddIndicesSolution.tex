\documentclass[12pt,letter]{article}

\usepackage[margin=1.0in]{geometry}

\begin{document}
\noindent\textsc{\large CS 314 Exam One Review --- Remove Odd Indices --- \textbf{Solution}}

\vspace{6pt}
\noindent\textbf{Array-based Lists}

\vspace{2pt}
\noindent Implement an instance method for the \texttt{GenericList} class which will 
remove all elements at odd indices. This method will then return the number of elements
removed from the list. None of the elements in the list will be \texttt{null}.

\vspace{4pt}
\noindent Complete the following method.
\begin{verbatim}
  // Removes all elements at odd indices
  // pre: none
  // post: returns number of elements removed from list
  public int removeOddIndices()
\end{verbatim}

\vspace{4pt}

\noindent Here are some example calls to \texttt{removeOddIndices()}:
\newline
\noindent \texttt{[0, 1, 2, 3, 4, 5].removeOddIndices()}$ \rightarrow $ \texttt{[0, 2, 4]}, returns 3
\newline
\noindent \texttt{["A", "B", "C", "D", "E"].removeOddIndices()} $ \rightarrow $ \texttt{["B", "D"]}, returns 3
\newline
\noindent \texttt{[314].removeOddIndices()} $\rightarrow$ \texttt{[314]}, returns 0
\newline
\noindent \texttt{[].removeOddIndices()} $\rightarrow$ \texttt{[]}, returns 0
\newline

\noindent Your method will be in the following \texttt{GenericList} class:

\begin{verbatim}
  public class GenericList<E>{
    private int size;
    private E[] con;
    // ...
  }

\end{verbatim}

\noindent \textbf{Do not use or assume there are any provided methods in the GenericList class.}

\noindent \textbf{Do not use any other Java classes or methods.}

\clearpage
\begin{verbatim}
  // Removes all elements at odd indices
  // pre: none
  // post: returns number of elements removed from list
  public int removeOddIndices() {
      int numToRemove = size / 2;
      if(numToRemove == 0)
          return 0;

      int indexToReplace = 1;
      //Loop through elements we want to keep
      for(int i = 2; i < size; i += 2){
          con[indexToReplace] = con[i];
          indexToReplace++;
      }
      //Null out elements past new size
      int newSize = size - numToRemove;
      for(int i = newSize; i < size; i++){
          con[i] = null;
      }

      //Set size variable
      size = newSize;
      return numToRemove;
  } 
\end{verbatim}

This is a relatively simple problem about removing elements from an array based list. However, when dealing 
with array based lists there are a lot of details which can be easy to forget. For instance, it is important
to update the size instance variable and to null out the elements past the new size of the list to prevent memory
leaks.

\end{document}
